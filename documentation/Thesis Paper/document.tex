\documentclass[12pt]{article}
\usepackage[a4paper,left=3cm,right=2cm,top=2.5cm,bottom=2.5cm]{geometry}
%opening
\title{Kickstarting Application Development\\ with Enterprise Java\\ A Spring Boot Case Study}
\author{Lennart Cockx\\Guangming Luo}

\begin{document}

\maketitle
\newpage
\tableofcontents
\newpage

\begin{abstract}
\noindent In this report we will explore various Java technologies and analyse what features they offer. The focus of this project is to find which frameworks are most suitable for quick prototyping of a business application.
Constraints and other requirements will be defined and will then be applied to a specific case study.
\end{abstract}

\section{Project Objective}
\subsection{Proposal}
This master thesis is commissioned by Faros, a Cronos group company specialized in the development of Java web applications and primarily focused on designing Rich Internet Applications. Development within Faros is done mainly with Vaadin, Spring, HTML5, ExtJS, JavaScript, JSP, JSF, NoSQL and more. This means that during this thesis, special attention is given to these technologies in regards to design choices. Naturally, comparisons with other possible frameworks will still be done and valid arguments will be set forward to confirm these decisions.
\\\\
The following case study was proposed by Faros: designing an online cash register that can be used in restaurants or at events. Existing cash register systems are expensive and for smaller establishments or temporary events, this investment is not worth it. The idea is to use existing devices from employees and customers to handle the ordering and payment process.
\\\\
This project is an opportunity for us to learn about the frameworks and technologies used for the development of rich, interactive web applications. Additionally, we want to explore the best methods to quickly build a web application from scratch. Some key aspects that we will look out for are: Documentation availability, speed of new project setup, customizability, integration with other frameworks and security features. When given the corresponding documentation, it should be possible to reproduce this application
\subsection{Concept}
At the start of the project we discuss the key aspects of this project with our coaches at Faros. Based on the requirements provided by them, combined with our own interests, we will define the general field we will be working in. 
\\\\
After we have a good overview of what aspects must be included in this project (or should be avoided), we can start deciding core features that will determine which technologies will be appropriate for this project. These choices are based on the aspects we think are needed for developing a web application based on the aforementioned requirements. This should give us a list of possible systems and frameworks that could potentially be used during this project.
\\\\
Now that we have a list of possible systems, we can start to think about the case study we want to develop. Once we know the general outline of the project, we can begin selecting the the appropriate tools and frameworks that fit this case study. This will shorten the list to technologies specifically applicable to this project. For certain frameworks it will be difficult to obtain a shorter list, as they might all comply with the requirements we set. In this case we will make a selection of items to keep the scope of the project realistic. Usually the most popular frameworks will be selected. If a deviating set is chosen this will be elaborated on in the corresponding chapter.
\\\\
Now that the tools are known, the standard procedure for application development will be followed. Relevant documentation will be constructed including a use case analysis, feature description (Nice-to-have versus Must-have), navigational models, hierarchical task analysis, UML diagrams and market research. After this we can start working on getting familiar with the chosen enterprise Java frameworks.
\\\\
As soon as we are sufficiently proficient with our chosen tools, we will start working on a case study where we use the knowledge we obtained to build a functional prototype for a realistic use case scenario. During the project we might come to the conclusion that certain choices we made were suboptimal or simply do not allow further progression. In such cases, we will adjust our trajectory and change to the appropriate tools. Any time such a decision is made we will mention it in this report and provide the matching argumentation as to why we made the switch.
\subsection{Project Requirements}
\subsubsection{Constraints}
\subsubsection{Design focus}

\section{Enterprise Java Research}
\subsection{Current Landscape}
\subsection{Comparison of technologies}
\subsubsection{General framework}
\subsubsection{Persistence providers}
\subsubsection{Security}
\subsubsection{Front-end and Styling}
\section{Case study: Online cash register}
\subsection{Introduction}
\subsection{Use Cases}
\subsection{Must-haves and nice-to haves}
\subsection{Technology choices}
\subsubsection{Application type}
\subsubsection{Framework choices}
\subsection{Design Choices}
\subsubsection{Client}
\subsubsection{Employee}
\subsubsection{Manager}
\subsubsection{Remarks and challenges}
\subsection{Major problems and their solutions}




blablabla \cite{greenwade93} \cite{antonov2015spring}


\newpage
\bibliographystyle{plain}
\bibliography{document}

\end{document}
